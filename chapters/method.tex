\chapter{A módszer leírása}
\label{ch:method}

A szemantikus reprenzentációs módszerek kutatása intenzíven felgyorsult az elmúlt évtizedben. Bár a szélesebb körben beszélt nyelvek esetében – például angol, kínai – számos technika és adathalmaz is elérhető, a kis és közepes nyelveknek egyelőre nélkülözniük kell ezeket. A probléma  feltehetőleg részben a kutatási terület újszerű jellegéből és a nagy tömegek igényeinek hiányából fakad. 

Ismereteim szerint magyar nyelven a tárgyalt kategóriák közül kizárólag szóbeágyazási modellek léteznek – mint a FastTex, Word2Vec és az ELMo – , továbbá a lehetséges tanítási feladatok is korlátozottak, így többnyire csak felügyelet nélküli tanítás elvégzése lehetséges. A diplomamunkám során megoldandó feladat egy mondat/paragrafus szintű nyelvi modell elkészítése, amely alapjául az előzményekben megismert módszerek szolgálnak. Továbbá olyan humán és autoannotált adathalmazok létrehozása és vizsgálata, melyeket a tanítási folyamathoz használok fel. Az így kapott előre tanított nyelvi modell reményeim szerint alkalmas lesz a későbbi NLP feladatokhoz szükséges finomhangolásra, továbbá a létrehozott adathalmazok és a tanításhoz használt algoritmusok más munkák segítségére is lehetnek.

A feladat megoldására szolgáló módszer alapvetően három részből áll: a bemeneti rétegből, a reprezentáció létrehozásához használt neurális hálóból és a modell tanításához definiált feladatokból, az ehhez alkalmazott fejekből. Mivel a szóalapú megoldások általában pontosabb eredményt mutatnak, mint a karakter, vagy szótöredék alapú modellek, így ebben az esetben is szavak kerülnek feldolgozásra.

(KÉP: architektúra)

Az implementációt Python nyelven végeztem a Tensorflow nevű könyvtár segítségével.

\section{Bemeneti réteg}

Ahogyan több módszer esetén is láthattuk, előfordulhat, hogy a neurális hálók bemenetére már eleve vektorizált formában érkeznek a token-ek. A feladat megoldásához használt architektúrában az input koordinálását egy bemeneti réteg végzi. Ezen réteg a felhasználó által konfigurálható attól függően, hogy az inputra a token-ek enkódolt formában érkeznek, vagy az algoritmus a számára megadható szóbeágyazási modellt használja. Ha a token-ek nem vektor formájában kerülnek a bemenetre, akkor a bemeneti réteg a token-ekhez rendelt egyedi azonosító számok szekvenciáját fogadja.

A tanítás során minden esetben a mélyháló számára megadott szóbeágyazási modellt használtam. A választott algoritmus a Word2Vec - CBOW volt. Eleinte, az implementáció alatt a wiki-hu adathalmazon tanított Word2Vec-et alkalmaztam, azonban az adatsor kis mérete miatt más megoldásokat kellett keresnem.  Ezek után több, az oscar\_hu halmazon tanított modellt is kipróbáltam. A végső és legjobb alternatíva egy kisebb, X méretű szótárral rendelkező, az oscar\_hu adatsoron tanított szóbeágyazás lett. A kiválasztási szempontok közé tartozott a pontosság és a memóriaigény, melyet a későbbiekben kifejtek.

(lookup kép)

A bemeneti réteg fix súlyokkal rendelkezik, tehát a tanulási folyamat során nem változtatja azokat. A mélyháló a számára átadott beágyazási mátrix elemei közül kikeresi az azonosítóknak megfelelő elemeket, majd továbbítja őket a kimenetre. Ezt a folyamatot \textit{embedding lookup}-nak nevezzük.

(ide még kéne valami)
(speciális tokenek a w2vben)

\section{A reprezentáció neurális hálója}

A rekurrens neurális hálók (RNN) használata a szemantikus reprezentációs modellek esetén gyakori technika. Míg a mesterséges neurális hálók csak önálló bemenet fogadására képesek, addig a rekurrens neurális hálók alkalmasak szekvenciális input feldolgozására is. Ilyen szekvencia például az idősori adat, vagy a szöveges adat is. A szekvenciális bemenetet az különbözteti meg önálló bemenettől, hogy a szekvenciális input elemei függenek egymástól, hatással lehetnek a szomszéd elemekre, több önálló input esetén ez a reláció nem érvényes.
A rekurrens neurális hálók képesek megtanulni az adatsor elemei közötti kapcsolatokat. Az RNN a tanulási folyamat során "emlékszik" az előző bemenetekből gyűjtött információkra, majd azok segítségével generálja a kimenetet/kimeneteket. A számítás során használt vektorokat nem csak az input súlyai befolyásolják, hanem a rekurrens háló rejtett állapotvektorai is. A rejtett állapot megtanulja a folytonos bemenet elemei közti függőségeket, majd minden tanítási lépés során frissül. Ennélfogva minden egyes bemeneti elem más és más műveleten esik át.

(RNN kép)

Bizonyos esetekben, ahol a múltból származó információ elegendő lehet a háló számára – például következő token generálása az előzőek függvényében – , az RNN jó opció lehet. Azonban olyan feladatok során, melyeknél fontos a bemeneti adatok kontextusa – például a nyelvi modellek – , más megoldásra van szükség. A BiRNN architektúra lényege, hogy az inputot két, egymással ellentétes irányú rekurrens háló olvassa. Az így kapott kimeneti vektorok páronkénti konkatenációja lesz a BiRNN output-ja.

(BIRNN kép)

Az RNN-ek legegyszerűbb formájának (\textit{Vanilla RNN}) azonban van egy nagy gyengesége, ami a hosszútávú információkat illeti. Gradiensnek hívjuk azokat az értékeket, melyeket a háló a súlyai frissítésére használ. Vanilla RNN esetén a visszaterjesztési művelet (\textit{backpropagation}) alatt annyira lecsökkenhetnek a túl kicsi gradiensek, hogy a hozzá tartozó rétegek megállnak a tanulásban. Ezt a problémát a \textit{vanishing gradients} problémának hívjuk.

Az LSTM (\textit{Long short-term memory}) architektúra megoldást nyújt a \textit{vanishing gradient} problémára. Az LSTM a megszokott hosszútávú memória mellé bevezeti a rövidtávú memóriát is. Olyan belső műveletei vannak, melyek képesek szabályozni az adott cellán belüli információáramlást. Ezen műveleteket kapuknak nevezzük. A kapuk eldönthetik, hogy mely információ lesz fontos a továbbiakban és melyiket lehet törölni. A módszer csak releváns információt enged a hosszútávú memóriába.

(LSTM kép)

(Írjak még az LSTMről?, kapuk, aktivációk részletesen)

Az általam a feladat megoldására választott architektúra az InferSent-ben kiváló eredményeket prezentáló BiLSTM + Max Pooling.

\subsection{A BiLSTM}
A BiLSTM egy kétirányú rekurrens architektúra (BiRNN), amely LSTM cellákat használ. Az NLP feladatok természetes nyelven írott szöveggel operálnak, így a rekurrens neurális háló az egyik alternatíva a változó hosszúságú szekvenciális adat feldolgozására. A kétirányú modell figyelembe veszi a feldolgozandó token kontextusát a tanulás során és eltárolja a sorrendi információkat is. Az LSTM cellák alkalmazása széles körben elterjedt technika, amely amellett, hogy képes kezelni az RNN gyengeségeit, az egyik jelenlegi legpontosabb megoldásnak bizonyul.

(kép bilstm pooling)

A sűrű rétegek tanítása közben az egyes neuronok között kialakulhatnak keresztfüggőségek, így túltanulhat a modellünk az adott adathalmazra. A \textit{dropout} egy olyan regularizációs technika, amely kikényszeríti, hogy az egyes neuronok önállóan tanuljanak, így véd a túltanulás ellen és a neurális háló is jobban fog generalizálni. A tanítási fázis során az összes iteráció, összes batch-e esetén minden neuron és hozzá tartozó aktiváció $1-p$ valószínűséggel véletlenszerűen kidobásra kerül. A teszt fázis alatt az összes neuron cselekvőképes, de az aktivációkat a helyes működés miatt $p$ állandóval szorozni kell. Bár a tanítási idő minden epoch során kevesebb lesz, a dropout körülbelül duplázza a konvergációhoz szükséges iterációk számát. A reprezentáció tanulására szolgáló neurális háló mindkét LSTM rétegére konfigurálható \textit{dropout}-ot alkalmaztam.

(kép dropout)

A BiLSTM réteg által generált szekvenciális kimenet egy \textit{pooling} rétegbe vezet.

\subsection{Pooling réteg}

A \textit{pooling} ötletét szintén a számítógépes képfeldolgozás ágazatától kölcsönözte az NLP. Míg konvolúciós rétegek esetén a \textit{feature map}-ek kisebb szegmensein elvégzendő a \textit{pooling} művelet, addig az NLP-ben vektorokra értendő.
A \textit{max pooling} réteg a kapott bemenet megadott tengelyei mentén választja ki a legnagyobb értékeket. Analóg módon a \textit{mean pooling} az átlagot veszi alapul.

A BiLSTM-ben található rejtett rétegek kimeneteinek páronkénti konkatenációján végzett pooling művelet képes kiválasztani a hasznos információt az egyes token-eket/rész szekvenciákat reprezentáló vektorokból. Az így kapott sorvektor lesz a bemeneti szekvencia végső reprezentációja.

A nyelvi modell neurális hálójának implementációja egy konfigurálható \textit{pooling} réteget tartalmaz, így a megoldás \textit{max} és \textit{mean pooling}-gal, továbbá pooling nélkül is képes dolgozni.

\subsection{Paraméterek és konfigurálhatóság}

A reprezentáció neurális hálójának implementációja során törekedtem a minél széleskörűbb konfigurálhatóságra, így elősegítve a könnyebb testreszabhatóságot és az újrafelhasználhatóságot. 

A felhasználó által állítható, modellre vonatkozó paraméterek a következők:
\begin{itemize}
	\item \textit{use\_embedding\_layer} : Alkalmazzon a háló bemeneti réteget, vagy vektorizált az input.
	\item \textit{word\_embedding\_dim} : A bemeneti szóbeágyazási vektorok mérete.
	\item \textit{num\_hidden} : A végső reprezentáció mérete, a rejtett LSTM rétegek méretének kétszerese.
	\item \textit{dropout\_keep\_prob} : Dropout valószínűség.
	\item \textit{pooling} : \textit{Pooling} fajtája. (Max, Mean, Nincs - ekkor szekvenciális kimenet)	
\end{itemize}

A neurális háló paraméterezhetősége lehetőséget biztosít arra, hogy az architektúra más típusú bemenettel, más célra is felhasználható legyen. Ezen konfigurációkon felül több, a tanítási folyamathoz kapcsolódó érték is állítható.

\section{Tanítás}

Bár a szemantikus reprezentációs modellek pontosságához jelentősen hozzájárul az alkalmazott architektúra teljesítménye, az elmúlt néhány év során mégis inkább a tanítóhalmazok és a tanítási módszerek felé irányult a figyelem.

A neurális hálóknak a korábbi hagyományos, feladatspecifikus tanítás alatt egyszerre kellett megérteni a dokumentumhalmaz nyelvét és megtanulni az adott feladathoz szükséges ismereteket. A \textit{transfer learning} technika segítségével ez a két folyamat különválasztható. Az előtanulás során feldolgozott nagy mennyiségű adat hatására a háló megragadja az adott dokumentumok nyelvi sajátosságait, így a finomhangolás alatt a modell koncentrálhat csak az adott feladatra. Az eredmény legtöbbször egy jobban működő reprezentáció.

Az előtanítási feladatok olyan jellegű kihívások elé állítják a reprezentáció neurális hálóját, amelyek során egy erős, általános tudást képes megszerezni. A probléma megoldásához implementált feladat a BERT előtanításához hasonló \textit{multi-task learning}. 

A \textit{multi-task learning} több feladaton egyszerre történő tanulást jelent, ami jobb generalizációra készteti a hálót. Az első feladat a \textbf{maszkolás}, amely az egyes szavak közötti szemantikai és szintaktikai jelentés ábrázolását támogatja. A második feladat a \textbf{következő mondat}, mely a mondatok közötti kohézió reprezentációját segíti.

Mivel az említett feladatok számára tanítóadatot bármely célnyelvű korpuszból könnyedén lehet generálni, ezért elméletileg az előtanítás a végtelenségig skálázható.

Bemenet hungarianWebcorpus -> hossz beállítás + grafikon

Bemenet MLM - masking, többi bemeneti cucc generálás
Bemenet NS random next sentence - balance

padding (min max sentence length)

Word2Vec 






\pagebreak




The training loss is the sum of the mean masked LM likelihood and mean next sentence prediction likelihood.

input generálás

padding

masking technika

+ speciális tokenek

LR decay

konfigs



\subsection{Maszkolás}

\subsection{Következő mondat}


\section{Vektorok generálása}