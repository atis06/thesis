\chapter{A módszer kiértékelése}
\label{ch:eval}

A modern szemantikus reprezentációs algoritmusok célja, olyan univerzális és általános modellek előállítása, amelyek bármely rendszerben képesek igazodni az adott kihívásokhoz és megfelelő pontossággal teljesíteni az eléjük tűzött feladatokat.

A nyelvi modellek teljesítményének mérése nem triviális feladat. Bár a publikációk során a szerzők többnyire saját módszerrel mérnek, vannak már meglévő komplex kiértékelési keretrendszerek – például SentEval \cite{senteval}, GLUE \cite{glue} – és az igény is egyre nagyobb ezekre. A teljesítmény mérésére szolgáló rendszerek segítségével egységes képet kaphatunk a módszerünk pontosságáról, illetve a csalás lehetősége is korlátozott. A kiértékelés közben a reprezentációs modelleknek olyan feladatok sorozatát kell megoldaniuk, mint a vélemény-polaritás, bináris érzelmi analízis, következtetés vizsgálat, szemantikus hasonlóság.

Mivel a munkám során tanított modellek mindegyike magyar nyelvű, így nem használhattam ezen megoldásokat. Szükségem volt egy saját kiértékelési feladat implementálására.

\section{Árukereső kommentek bináris érzelmi analízise}



%\pagebreak

%Szükség volt a modellek hatékonyságának/pontosságának kimérésére valamely szempont alapján. Ehhez kellett egy feladat, melyet magamnak kellett implementálni. Ez a feladat kellően nehéz kell hogy legyen. Árukereső adathalmaz erre pontosan jó, egyszerűen interpretálható végeredmény -> jó összehasonlítási alap

%saját halmaz - nem túl kiterjedt, tehát nem a nyelvi modell egészét méri de adott feladatra jó

%metrikák: tanítási accuracy, árukereső accuracy, mátrix, roc

%w2v vektorok átlaga - baseline: w2v_lg, w2v_lg_normed, w2v_sm, w2v_sm_normed
%teljes szövek kódolása

%saját modellek és részletes leírásaik: lstm_1024, lstm_1024_normed, lstm_4096
%201 token levágása és kódolása

%arukereso auto annotált/humán annotált?







