\chapter{Bevezetés} % Introduction
\label{ch:intro}

A természetesnyelv-feldolgozás (NLP) a mesterséges intelligencia azon részterülete, amely az emberi eredetű beszélt és írott nyelvből történő információkinyeréssel és ezen tudás felhasználásával foglalkozik. A szemantikus reprezentáció az NLP intenzíven kutatott témaköre, amely algoritmusai képesek a természetes nyelven írott szövegek és szövegrészletek numerikus ábrázolására. A módszerek alapja, hogy a szavakat, vagy szavak listáját leképezzék valamely vektortérbe azok szemantikai tartalma alapján. 

Az így kapott vektoroknak számos felhasználási módja létezik, például információ visszakeresés, dokumentum összegzés, chatbot-ok implementálása, gépi fordítás, stb. Napjainkban a legjobb eredményeket az ezeken a területeken kutató és fejlesztő óriáscégek által publikált módszerek érik el, de a hasonló technikák akadémiai körökben is nagy figyelmet kapnak.

A modern reprezentációs módszerek meghatározó tényezői az alapjukként szolgáló neurális háló és a tanításukra használt feladatok, adathalmazok.
Bár léteznek többnyelvű reprezentációs modellek is, a meglévő technikák nagy részéről kijelenthető, hogy az nyelvfüggő. A nyelvfüggőség azt jelenti, hogy egy adott modell csak olyan nyelvű problémák esetén alkalmazható, amilyen nyelven tanították azt.

A friss eredmények azt mutatják, hogy címkézett adatokon történő felügyelt tanítás után modellünk magasabb teljesítményre lehet képes. Ez a tény problémát jelenthet a kevésbé beszélt nyelvek esetén, ahol csak elvétve, vagy egyáltalán nem léteznek ilyen tanítóadatok. A kevésbé populáris nyelveken való tanítás során jellemzően csak a nem felügyelt tanulás eszköztárából választhatunk.

A magyar egy nem túl széles körben beszélt nyelv, így a nyelvi modellek tanításához használható források is limitáltak. Diplomamunkám célja a meglévő módszerek vizsgálata, majd ezen tudás alapján olyan tanítóhalmazok létrehozása és technikák implementálása, amelyek alkalmasak lehetnek a kis és közepes nyelvek – így a magyar nyelv – szemantikai és szintaktikai tulajdonságainak ábrázolására. 
