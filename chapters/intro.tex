\chapter{Bevezetés} % Introduction
\label{ch:intro}

A természetesnyelv-feldolgozás (NLP) a mesterséges intelligencia azon részterülete, amely az emberi eredetű beszélt és írott nyelvből történő információkinyeréssel és ezen tudás felhasználásával foglalkozik. A szemantikus reprezentációk az NLP intenzíven kutatott témaköre, amely algoritmusai képesek a természetes nyelven írott szövegek és szövegrészletek numerikus ábrázolására. A módszerek alapja, hogy a szavakat, vagy szavak listáját leképezzék valamely vektortérbe azok szemantikai tartalma alapján.

Az így készült vekto

\pagebreak

AI -> NLP
mi az a semantic representation?
milyen elméleti problémát old meg?
nagy cégek, gyorsan fejlődik
thought vectors
hol használják?
translation
nyelvfüggő
kevés adat és módszer magyarul
mi a célom? 
adat, tanítási lehetőség, algoritmus vizsgálat
finnugor relációt el lehetne sütni
esetleg matematikai formalizmusa (szólista -> vektor)

