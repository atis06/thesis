\chapter{Összegzés} % Conclusion
\label{ch:sum}

Diplomamunkám végeredménye egy mondat- és paragrafusszintű szemantikus reprezentációs algoritmus, amely képes leképezni a magyar nyelven írt mondatokat a szemantikus térbe.
 
(adatok + háló + feladatok)

Jó eséllyel a módszer alkalmazható más kis és közepes nyelvre is.



A munka során kipróbáltam több adathalmazt is, melyekkel a Word2Vec modelleket és a saját modellt tanítottam,

 továbbá létrehoztam egy magyar nyelvű adathalmazt is, amelyen érzelmi tartalom szerinti bináris klasszifikációt lehet végezni.
 
 kimértem több osztályozó algoritmussal a teljesítményeket(Saját mérési adathalmaz + baseline módszer)








%TODO: EREDMÉNYEK
kísérleteztem és tanítottam számos modellt különböző paraméterekkel -> legjobb


jövőben: maszkolt UNK-ok 0 súllyal, finomhangolás, stacked bilstm, GloVe, gru,  értelmező kéziszótár, mean pooling

új terület -> fel kell térképezni, más más beállításokat alkalmazni, 

magyar NLP ezen ágát eg kicsit katalizálja és a benchmarkot is fel tudják használni

köszönet